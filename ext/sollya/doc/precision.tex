\subsection{precision}
\label{labprecision}
\noindent Name: \textbf{precision}\\
returns the precision necessary to represent a number.\\
\noindent Usage: 
\begin{center}
\textbf{precision}(\emph{x}) : \textsf{constant} $\rightarrow$ \textsf{integer}\\
\end{center}
Parameters: 
\begin{itemize}
\item \emph{x} is a dyadic number.
\end{itemize}
\noindent Description: \begin{itemize}

\item \textbf{precision}(x) is by definition $\vert x \vert$ if x equals 0, NaN, or Inf.

\item If \emph{x} is not zero, it can be uniquely written as $x = m \cdot 2^e$ where
   $m$ is an odd integer and $e$ is an integer. \textbf{precision}(x) returns the number
   of bits necessary to write $m$ in binary (i.e. $\lceil \log_2(m) \rceil$).
\end{itemize}
\noindent Example 1: 
\begin{center}\begin{minipage}{15cm}\begin{Verbatim}[frame=single]
> a=round(Pi,20,RN);
> precision(a);
19
> m=mantissa(a);
> ceil(log2(m));
19
\end{Verbatim}
\end{minipage}\end{center}
See also: \textbf{mantissa} (\ref{labmantissa}), \textbf{exponent} (\ref{labexponent}), \textbf{round} (\ref{labround})
