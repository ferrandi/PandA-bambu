\subsection{diff}
\label{labdiff}
\noindent Name: \textbf{diff}\\
differentiation operator\\
\noindent Usage: 
\begin{center}
\textbf{diff}(\emph{function}) : \textsf{function} $\rightarrow$ \textsf{function}\\
\end{center}
Parameters: 
\begin{itemize}
\item \emph{function} represents a function
\end{itemize}
\noindent Description: \begin{itemize}

\item \textbf{diff}(\emph{function}) returns the symbolic derivative of the function
   \emph{function} by the global free variable.
    
   If \emph{function} represents a function symbol that is externally bound
   to some code by \textbf{library}, the derivative is performed as a symbolic
   annotation to the returned expression tree.
\end{itemize}
\noindent Example 1: 
\begin{center}\begin{minipage}{15cm}\begin{Verbatim}[frame=single]
> diff(sin(x));
cos(x)
\end{Verbatim}
\end{minipage}\end{center}
\noindent Example 2: 
\begin{center}\begin{minipage}{15cm}\begin{Verbatim}[frame=single]
> diff(x);
1
\end{Verbatim}
\end{minipage}\end{center}
\noindent Example 3: 
\begin{center}\begin{minipage}{15cm}\begin{Verbatim}[frame=single]
> diff(x^x);
x^x * (1 + log(x))
\end{Verbatim}
\end{minipage}\end{center}
See also: \textbf{library} (\ref{lablibrary}), \textbf{autodiff} (\ref{labautodiff}), \textbf{taylor} (\ref{labtaylor}), \textbf{taylorform} (\ref{labtaylorform})
